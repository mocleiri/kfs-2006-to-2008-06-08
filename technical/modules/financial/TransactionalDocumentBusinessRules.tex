\documentclass[12pt]{article}
\usepackage{hyperlatex}
\usepackage[dvips]{graphicx}

\htmltitle{Business Rules Validation for TransactionalDocuments}
\htmladdress{przybylskil@arizona.edu}
\htmldirectory{TransactionalDocumentBusinessRules}
\htmlname{index}
\htmlcss{../css/hlatex.css}

\newcommand{\sourcehead}[1]
 {\texorhtml{\sf}{\xml*{span class='cembedhead'}}#1\texorhtml{\rm}{\xml*{span class='cembedhead'}}}
\newcommand{\source}[1]
 {\texorhtml{\sf}{\xml*{span class='cembed'}}#1\texorhtml{\rm}{\xml*{span class='cembedhead'}}}
\newcommand{\sourcefrag}[1]
{\texorhtml{\sf}{\xml*{div class='source'}}#1\texorhtml{\rm}{\xml*{/div}}}

\title{Business Rules Validation for \sourcehead{TransactionalDocument}s}
\author{Leo Przybylski}

\begin{document}
  \maketitle
  \tableofcontents
  \listoffigures
  
  \section{Business Rules Validation Framework}
  \texonly{\indent} If you are already relatively familiary with the validation
  framework, then you might as well just skip this section. You only need to 
  read this if you need a brief refresher or just skim over it for a basic
  understanding.

  Whenever an action is taken on a \source{TransactionalDocument}, an event
  is created and handled by the \source{KualiRulesService}.
  
  \subsection{\sourcehead{AddAccountingLineEvent}}
  This event occurs when an \source{AccountingLine} is added.

  \subsection{\sourcehead{GeneratePendingEntriesEvent}}
  This event occurs when pending entries are generated for the general ledger.
  
  \subsection{\sourcehead{RouteDocumentEvent}}
  This event occurs when a \source{TransactionalDocument} is submitted for 
  routing. Other events are really implicit to this one.
  
  \subsection{\sourcehead{SaveDocumentEvent}}
  This event occurs when a \source{TransactionalDocument} is saved. To my
  knowledge, only the \source{BursarControlScreen} is really going to use this
  because it is never routed, so rules must be handled during save. Usually,
  \source{TransactionalDocument} instances should be saved without any kind
  of validation.

  \subsection{\sourcehead{KualiRulesService}}
    \texorhtml{\includegraphics[scale=0.60]{Diagrams/KualiRuleService_class.eps}}{
    \htmlimg{../Diagrams/InternalBillingDocument_class.png}
  }
    
  This is just an aggregation class for validating business rules against
  business objects. \source{KualiRulesService} is notified when an
  \source{Event} is to be handled. The \source{applyRules()} method does
  all the work dealing with aggregation.

  \subsection{\sourcehead{KualiRulesService}}
  \texorhtml{\includegraphics[scale=0.60]{Diagrams/KualiRuleService_Relationships.eps}}{
    \htmlimg{../Diagrams/InternalBillingDocument_class.png}
  }  

  \subsubsection{\source{applyRules()}}
  There is an \source{applyRules()} method for every valid \source{Event}. For
  example,
  
\sourcefrag{\begin{quote}public boolean applyRules(AddAccountingLineEvent event)\end{quote}}
  
  The above method describes an \source{applyRules()} definition that handles
  an \source{AddAccountingLineEvent}. It returns a \source{boolean} primitive
  type. This type refers to whether the rule passed when it was applied to the
  \source{Document}.

  \source{GeneratePendingEntriesRule} is the only rule that requires a
  \source{TransactionalDocument} making it specifically a
  \source{TransactionalDocument} rule. The other rules reference a
  \source{Document}, so they are not.
  
  \subsection{\sourcehead{AddAccountingLineRule}}
  \texorhtml{\includegraphics[scale=0.60]{Diagrams/AddAccountingLineRule_Class.eps}}{
    \htmlimg{../Diagrams/InternalBillingDocument_class.png}
  }  

  This rule is used for \source{AddAccountingLineEvent} occurrences.
  
  \subsubsection{\source{processAddAccountingLine()}}
  \sourcefrag{\begin{quote}public boolean processAddAccountingLine(Document document, AccountingLine accountingLine)\end{quote}}

  \subsection{\sourcehead{GeneratePendingEntriesRule}}
  \texorhtml{\includegraphics[scale=0.45]{Diagrams/GeneratePendingEntriesRule_Class.eps}}{
    \htmlimg{../Diagrams/InternalBillingDocument_class.png}
  }  

  This rule is used for \source{GeneratePendingEntriesEvent} occurrences.

  \subsubsection{\source{processGeneratePendingEntries()}}
  \sourcefrag{\begin{quote}public boolean processGeneratePendingEntries(TransactionalDocument transactionalDocument, AccountingLine accountingLine, GeneralLedgerPendingEntrySequenceHelper sequenceHelper)\end{quote}}

  \subsection{\sourcehead{RouteDocumentRule}}
  \texorhtml{\includegraphics[scale=0.60]{Diagrams/RouteDocumentRule_Class.eps}}{
    \htmlimg{../Diagrams/InternalBillingDocument_class.png}
  }  

  This rule is used for \source{RouteDocumentEvent} occurrences.

  \subsubsection{\source{processRouteDocument()}}
  \sourcefrag{\begin{quote}public boolean processRouteDocument(Document document)\end{quote}}

  \subsection{\sourcehead{SaveDocumentRule}}
  \texorhtml{\includegraphics[scale=0.60]{Diagrams/SaveDocumentRule_Class.eps}}{
    \htmlimg{../Diagrams/InternalBillingDocument_class.png}
  }  

  This rule is used for \source{SaveDocumentEvent} occurrences.

  \subsubsection{\source{processSaveDocument()}}
  \sourcefrag{\begin{quote}public boolean processSaveDocument(Document document)\end{quote}}

  \section{\sourcehead{TransactionalDocument } \emph{Business Objects}}
  To reduce code duplication and add orthogonality to the framework, the
  \source{TransactionalDocumentRuleBase} was created. It is most likely that
  this class will be used when creating \emph{business rules} because most of
  the \emph{business rules} interfaces promote use of 
  \source{TransactionalDocument} which is what 
  \source{TransactionalDocumentRuleBase} is intended to supply base rules 
  functionality for.

  Simply put, if you're reading this, you want to use the 
  \source{TransactionalDocumentRuleBase}.

  This section is about how to use \source{TransactionalDoucmentRuleBase} 
  effectively. \source{TransactionalDocumentRuleBase} covers most of what 
  one would ever need to do as far as \emph{Kuali Business Rules}, but not all 
  \source{TransactionalDocument} definitions are the same. Many, of course,
  have their own special needs. Ways to fulfill the rules for those special
  needs as well as making the most out of 
  \source{TransactionalDocumentRuleBase} is what this section is all about.

  There are examples of simple implementations of 
  \source{TransactionalDocumentRuleBase}, and also some examples
  of how to properly override methods of \source{TransactionalDocumentRuleBase}
  for special needs.
  

  \subsection{\sourcehead{TransactionalDocumentRuleBase} Abstraction}
  \texorhtml{\includegraphics[scale=0.50]{Diagrams/TransactionalDocumentRuleBase_Relationships.eps}}{
    \htmlimg{../Diagrams/InternalBillingDocument_class.png}
  }  

  \source{TransactionalDocumentRuleBase} is a base implementation of
  the \emph{Business Rules Validation Framework} interfaces
  \source{AddAccountingLineRule}, \source{GeneratePendingEntriesRule},
  \source{RouteDocumentRule}, and \source{SaveDocumentRule}.

  \subsubsection{Under the Hood}
  \texorhtml{\includegraphics[scale=0.50]{Diagrams/KualiTransactionalDocumentActionBase_Sequence.eps}}{
    \htmlimg{../Diagrams/InternalBillingDocument_class.png}
  }

  The above shows how an \source{AddAccountingLineEvent} is handled using
  an \source{AddAccountingLineRule}. Notice \source{AccountingLine} instances
  are created and handled throught the \emph{Struts} Framework.
  \newpage

  \texorhtml{\includegraphics[scale=0.50]{Diagrams/KualiRuleServiceRouteDocument_Sequence.eps}}{
    \htmlimg{../Diagrams/InternalBillingDocument_class.png}
  }
  
  This model shows how \source{RouteDocumentEvent} instances are handled. 
  \source{SaveDocumentEvent} instances are handled almost exactlyt the same,
  just with a different rule.
  \newpage

  \texorhtml{\includegraphics[scale=0.40]{Diagrams/KualiRuleServiceGLPE_Sequence.eps}}{
    \htmlimg{../Diagrams/InternalBillingDocument_class.png}
  }

  As illustrated above, the \source{DocumentService} triggers 
  \source{GeneralLedgerPendingEntry} instances to be generated upon routing.
  The called \source{GeneralLedgerPendingEntryService} handles the 
  \source{GeneratePendingEntriesEvent} by passing it to the 
  \source{KualiRuleService}. You may recognize the \source{validateDocument()}
  method that is called early on. This is the same one as illustrated before
  that handles the \source{RouteDocumentEvent}.


  \subsection{Getting Started with \sourcehead{TransactionalDocumentRuleBase}}
  \subsubsection{Embedded Codes}
  \texorhtml{\includegraphics[scale=0.35]{Diagrams/TransactionalDocumentRuleBase_Inner.eps}}{
    \htmlimg{../Diagrams/InternalBillingDocument_class.png}
  }  

  A lot of work is already done. Code definitions are embedded into the
  \source{TransactionalDocumentRuleBase} as inner classes.
  \begin{itemize}
    \item \source{TransactionalDocumentRuleBase.CONSOLIDATED\_OBJECT\_CODE}
    \item \source{TransactionalDocumentRuleBase.FUND\_GROUP\_CODE}
    \item \source{TransactionalDocumentRuleBase.GENERAL\_LEDGER\_PENDING\_ENTRY\_CODE}
    \item \source{TransactionalDocumentRuleBase.OBJECT\_CODE}
    \item \source{TransactionalDocumentRuleBase.OBJECT\_LEVEL\_CODE}
    \item \source{TransactionalDocumentRuleBase.OBJECT\_SUB\_TYPE\_CODE}
    \item \source{TransactionalDocumentRuleBase.OBJECT\_TYPE\_CODE}
    \item \source{TransactionalDocumentRuleBase.SUB\_FUND\_GROUP\_CODE}
  \end{itemize}

  \subsubsection{Methods to Override}
  \texorhtml{\includegraphics[scale=0.55]{Diagrams/TransactionalDocumentRuleBase_Overrides.eps}}{
    \htmlimg{../Diagrams/InternalBillingDocument_class.png}
  }
  The above is a class model view of \source{TransactionalDocumentRuleBase}
  illustrating which methods are suggested to be overidden for customization
  of \source{TransactionalDocumentRuleBase}.

  \subsubsection{Helper Methods}
  \texorhtml{\includegraphics[scale=0.65]{Diagrams/TransactionalDocumentRuleBase_Helpers.eps}}{
    \htmlimg{../Diagrams/InternalBillingDocument_class.png}
  }  

  Helper methods exist to easily analyzing the state of
  \source{TransactionalDocument} and \source{AccountingLine} instances. There
  are factory methods for generating \source{GeneralLedgerPendingEntry} 
  instances, and there are also embedded default rule definitions for handling
  each of the business rule classifications (\source{AddAccountingLineRule}, 
  \source{GeneratePendingEntriesRule}, \source{RouteDocumentRule}, or 
  \source{SaveDocumentRule})
  
  \subsubsection{Implicit Rules}
  It's important to know what the implicit rules are because these define the
  default behavior of the \source{TransactionalDocumentRuleBase}.

  \subsection{\sourcehead{NonCheckDisbursementDocumentRule}: Case Study of a Basic Implementation}
  This is a good example of how \source{TransactionalDocumentRuleBase} covers
  a majority of the rules for a \source{TransactionalDocument}. The only 
  method that required overriding was:

  \sourcefrag{\begin{quote}protected boolean processCustomAddAccountingLineBusinessRules(TransactionalDocument document, AccountingLine accountingLine )\end{quote}}

  The overridden method simply validates the \source{AccountingLine} by
  comparing characteristics with contents of statically defined 
  \source{TreeSet} instances.
\begin{verbatim}
private static final Set _invalidObjectTypes = new TreeSet();
private static final Set _invalidSubTypes = new TreeSet();
private static final Set _invalidSubFundGroupTypes = new TreeSet();
...
...
static {
    _invalidObjectTypes.add(OBJECT_TYPE_CODE.INCOME_NOT_CASH);
    _invalidObjectTypes.add(OBJECT_TYPE_CODE.EXPENSE_NOT_EXPENDITURE);

    _invalidSubTypes.add(OBJECT_SUB_TYPE_CODE.BUDGET_ONLY);
    _invalidSubTypes.add(OBJECT_SUB_TYPE_CODE.CASH);
    _invalidSubTypes.add(OBJECT_SUB_TYPE_CODE.SUBTYPE_FUND_BALANCE);
    _invalidSubTypes.add(OBJECT_SUB_TYPE_CODE.HOURLY_WAGES);
    _invalidSubTypes.add(OBJECT_SUB_TYPE_CODE.PLANT);
    _invalidSubTypes.add(OBJECT_SUB_TYPE_CODE.SALARIES);
    _invalidSubTypes.add(OBJECT_SUB_TYPE_CODE.VALUATIONS_AND_ADJUSTMENTS);
    _invalidSubTypes.add(OBJECT_SUB_TYPE_CODE.RESERVES);
    _invalidSubTypes.add(OBJECT_TYPE_CODE.MANDATORY_TRANSFERS);
    _invalidSubTypes.add(OBJECT_SUB_TYPE_CODE.FRINGE_BEN);
    _invalidSubTypes.add(OBJECT_SUB_TYPE_CODE.COST_RECOVERY_EXPENSE);
    _invalidSubFundGroupTypes.add(SUB_FUND_GROUP_CODE.RENEWAL_AND_REPLACEMENT);
}
...
...
    retval &= !getInvalidObjectTypes()
        .contains( getObjectType() );
    retval &= !getInvalidSubTypes()
        .contains( getObjectSubType() );
    retval &= !getInvalidSubFundGroupTypes()
        .contains( getSubFundGroupCode() );
...
...
\end{verbatim}
The method implementation is generic. What sets it apart from other rules are
the static definitions.
\begin{verbatim}
...
...
static {
    _invalidObjectTypes.add(OBJECT_TYPE_CODE.INCOME_NOT_CASH);
    _invalidObjectTypes.add(OBJECT_TYPE_CODE.EXPENSE_NOT_EXPENDITURE);

    _invalidSubTypes.add(OBJECT_SUB_TYPE_CODE.BUDGET_ONLY);
    _invalidSubTypes.add(OBJECT_SUB_TYPE_CODE.CASH);
    _invalidSubTypes.add(OBJECT_SUB_TYPE_CODE.SUBTYPE_FUND_BALANCE);
    _invalidSubTypes.add(OBJECT_SUB_TYPE_CODE.HOURLY_WAGES);
    _invalidSubTypes.add(OBJECT_SUB_TYPE_CODE.PLANT);
    _invalidSubTypes.add(OBJECT_SUB_TYPE_CODE.SALARIES);
    _invalidSubTypes.add(OBJECT_SUB_TYPE_CODE.VALUATIONS_AND_ADJUSTMENTS);
    _invalidSubTypes.add(OBJECT_SUB_TYPE_CODE.RESERVES);
    _invalidSubTypes.add(OBJECT_TYPE_CODE.MANDATORY_TRANSFERS);
    _invalidSubTypes.add(OBJECT_SUB_TYPE_CODE.FRINGE_BEN);
    _invalidSubTypes.add(OBJECT_SUB_TYPE_CODE.COST_RECOVERY_EXPENSE);
    _invalidSubFundGroupTypes.add(SUB_FUND_GROUP_CODE.RENEWAL_AND_REPLACEMENT);
}
...
...
\end{verbatim}

The static definitions vary based-on the rules. Also, you can easily see the 
value of the embedded codes.

  \subsection{\sourcehead{AuxillaryVoucherDocumentRule}: Case Study of a Customized Implementation}
  Sometimes, a \source{TransactionalDocument} is just too different from other
  \source{Document} definitions and so are the rules. In order to get the 
  desired rules functionality, \source{AuxilaryVoucherDocumentRule} needed to
  override a number of methods.

  I won't go into detail the source code, but I will briefly describe why
  each was overridden.

  \subsubsection{\sourcehead{getPendingEntryAmountFor()}}
  \subsubsection{\sourcehead{customizeExplicitPendingEntry()}}
  \subsubsection{\sourcehead{customizeOffsetPendingEntry()}}
  \subsubsection{\sourcehead{processCustomGeneratePendingEntriesBusinessRules()}}
  \subsubsection{\sourcehead{isDebit()}}
  \subsubsection{\sourcehead{processCustomAddAccountingLineBusinessRules()}}
  \subsubsection{\sourcehead{processCustomRouteDocumentBusinessRules()}}

  In short, \source{AuxillaryVoucherDocumentRule} disagreed with the implicitly
  defined rules of \source{TransactionalDocumentRuleBase}.

\end{document}
