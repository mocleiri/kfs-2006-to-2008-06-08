\begin{ifhtml}
    \begin{s5slide}
        \section{What is the Lookup Framework?}
            \begin{itemize}
                \item Allows users to look up Business Objects (defined in DD files) from:
                \begin{itemize}
                    \item Maintenance Documents
                    \item Transactional Documents
                    \item Lookup screens
                    \item Just about anywhere in Kuali, given the right configuration
                \end{itemize}
                \item These BOs values (primary keys) are then passed back to the document the user was working on
            \end{itemize}
    \end{s5slide}
    \begin{s5slide}
        \section{How does the Lookup Framework work?}
        Alot of it is automatic, but can require some setup
            \begin{itemize}
                \item Not all BOs are lookupable
                \begin{itemize}
                    \item If they aren't defined through DD then you won't be able to look them up
                    \item If you don't have references for all of the keys then you can't look them up (no place to store primary key data)
                \end{itemize}
                \item To determine if a BO reference is lookupable KFS must be able to:
                \begin{itemize}
                    \item First, find its entry in the DD
                    \item Second, are lookups turned off for this BO?
                    \item Finally, make sure that all the primary keys match and are storable
                \end{itemize}
            \end{itemize}
            \begin{s5notes}
                Walk through some of the code for determining Lookupability
            \end{s5notes}
    \end{s5slide}
    \begin{s5slide}
        \section{Once it determines it can\\ be looked up what happens?}
            \begin{itemize}
                \item There are two sides to every lookup:
                \begin{itemize}
                    \item Document doing the lookup
                    \item The return (values being returned)
                \end{itemize}
                \item Since KFS is mostly URL driven (not much stored in the session) we transform the request and return into the URL of the Lookup itself
            \end{itemize}
            \begin{s5notes}
                Show the HTML code for a common lookup
            \end{s5notes}
    \end{s5slide}
    \begin{s5slide}
        \section{What if I want to\\ return multiple values?}
            \begin{itemize}
                \item Multiple value lookups are used on Globals (remember we mentioned those briefly?)
                \begin{itemize}
                    \item Globals operate on many BOs at once and you need an easy way for users to select multiple items at once to return to the Global document
                \end{itemize}
                \item So, the Lookup will then take those temporary (selected) values and put them into a temporary space in the database until the user returns to the originating document
                \item But the same principle of Lookups is still the same
            \end{itemize}
    \end{s5slide}
    \begin{s5slide}
        \section{Introducing BusinessObjectMetaDataService}
            This class helps us with the MetaData of a BO class (including information from the DD as well as OJB).
            \begin{s5notes}
                show BOMDServiceImpl
            \end{s5notes}
    \end{s5slide}
    \begin{s5slide}
        \section{Leo's stuff on Inquiry}
            
    \end{s5slide}
\end{ifhtml}