\begin{ifhtml}
    \begin{s5slide}
        \section{Kuali Stack Overview}
        \begin{ifhtml}
            \xml{img style="align:center" src="../Diagrams/KNS-Components.png"}
        \end{ifhtml} 
    \end{s5slide}
    \begin{s5slide}
        \section{What are we using?}
        \begin{ifhtml}
            \begin{itemize}
                \item Struts for our view layer
                \item Spring for our dependency injection layer
                \item OJB for our ORM (Object-Relational Mapping) Layer
            \end{itemize}
            \begin{itemize}
                \item Our custom pieces
                \begin{itemize}
                    \item Service Layer (driven through Spring)
                    \begin{itemize}
                        \item Document Service
                        \item Rules Service
                        \item Document Authorizer Service
                        \item Persistence Service
                        \item DataDictionary Service
                    \end{itemize}
                    \item Documents
                    \item Business Objects (BOs)
                    \item Application Parameters (rules helpers - db backed rules)
                    \item Workflow
                \end{itemize}
            \end{itemize}
        \end{ifhtml} 
    \end{s5slide}
    \begin{s5slide}
        \section{How do these pieces fit together?}
        \begin{ifhtml}
            \begin{itemize}
                \item Struts
                \begin{itemize}
                    \item View layer used for interaction with the user
                    \item controls formatting of data back and forth to Web using formatters 
                \end{itemize}
                \item Document
                \begin{itemize}
                    \item Primary interaction point for dealing with the system
                    \item Contains all the BOs that need to be persisted to the system
                    \item Accessed through Struts using \code{KualiForm}
                \end{itemize}
                \item DataDictionary
                \begin{itemize}
                    \item Used to retrieve Metadata about documents and BOs
                    \item Contains UI information as well as other data used to process documents and BOs
                \end{itemize}
            \end{itemize}
        \end{ifhtml} 
    \end{s5slide}
    \begin{s5slide}
        \section{Service Oriented Architecture}
        \begin{ifhtml}
            \begin{itemize}
                \item Kuali uses SOA to reduce inter-dependency between implementations
                \begin{itemize}
                    \item Easier to swap out \code{DocumentAuthorizerService} with your own implementation
                    \item Or replace the \code{RulesService} with your own custom implementation that uses non-code solutions (Drools, JESS, etc.)
                \end{itemize}
                \item Spring is used to control the inter-dependencies between service layers
                \item We will review Spring more this afternoon
            \end{itemize}
        \end{ifhtml} 
    \end{s5slide}
    \begin{s5slide}
        \section{Kuali Domain Model}
        \begin{ifhtml}
            \begin{itemize}
                \item Domain Model consists of:
                \begin{itemize}
                    \item Business Object (\code{ObjectCode.java})
                    \item DataDictionary file (\code{ObjectCode.xml})
                    \item OJB Mapping for BO (\code{OJB-repository.chart.xml})
                    \item DAO - Data Access Object (not all BOs have DAOs)
                    \item We will cover the domain model in more detail tomorrow morning
                    
                \end{itemize}
            \end{itemize}
        \end{ifhtml}
        \begin{s5notes}
            Show the files for the different pieces of the Domain Model
        \end{s5notes} 
    \end{s5slide}
    
\end{ifhtml}