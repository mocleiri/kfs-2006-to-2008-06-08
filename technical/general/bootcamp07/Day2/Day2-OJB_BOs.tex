\begin{ifhtml}
    \begin{s5slide}
        \section{OJB and Business Objects}
        \begin{itemize}
          \item Domain Model Review
          \item OJB Mapping Review
          \item BusinessObjects
          \item OJB, Spring, and DAO's ... again
        \end{itemize}
        \begin{s5notes}
        \end{s5notes}
    \end{s5slide}
    
    \begin{s5slide}
      \section{Domain Model Review}
      \begin{itemize}
        \item Two distinctions (Rich and Simple) Domain Models
        \item Domain Model represents an architecture where business objects and business rules are 
          abstracted from functionality and persistence layers
        \item Kuali uses Spring to create a service layer to act as interfaces to business objects and
          rules
        \item Kuali uses OJB to represent an abstraction of the persistence layer
        \item Kuali uses something called a \xml{u}PersistenceBroker\xml{/u} for the service layer to
          contact the persistence layer
      \end{itemize}
      \begin{s5notes}
        \begin{itemize}
        \item Two distinctions (Rich and Simple) Domain Models
        \item Domain Model represents an architecture where business objects and business rules are 
          abstracted from functionality and persistence layers
        \item Kuali uses Spring to create a service layer to act as interfaces to business objects and
          rules
        \item Kuali uses OJB to represent an abstraction of the persistence layer
        \item Kuali uses something called a \xml{u}PersistenceBroker\xml{/u} for the service layer to
          contact the persistence layer
        \end{itemize}
      \end{s5notes}
    \end{s5slide}

    \begin{s5slide}
      \section {OJB Mapping Review}
      \begin{itemize}
        \item Like Spring and JUnit, supports programming to interface.
        \item Mappings try to handle RDBMS table structure relationships and object-oriented hierarchy relationships
           at the same time. Tricky.
         \item In a Simple Domain Model, there mostly mappings for 1-to-1 table/java object.
         \item Kuali uses the Optimistic Locking paradigm supported by OJB
      \end{itemize}
      
      \begin{s5notes}
        \begin{itemize}
          \item OJB allows interface mappings. This means that a table can be mapped to a Java interface.
          \item OJB can handle inheritence. That means more than one class/interface using the same table, 
            or multiple tables.
          \item Not advocating OJB, so much as the power of ORM's and Spring PersistenceBroker.
          \item Kuali uses the Simple Domain Model or Rich Domain Model?
          \item OJB supports different mechanisms for locking tables, but Kuali uses optimistic locking which
            is a software implementation. This means that it is not provided natively by the RDBMS. OJB handles 
            optimistic locking similar to CVS (by using a version number.) The way it works is each table has a
            ``versionNumber.'' When the record is updated, it compares version numbers. If two updates happen
            simultaneously, they will have different version numbers. Only the transaction with the most recent
            version number can be updated. The other transaction will be thrown out, forgotten, lost in the ether.
        \end{itemize}
      \end{s5notes}
    \end{s5slide}

    \begin{s5slide}
      \section{Business Objects}
      \begin{itemize}
        \item Commonly referred as a POJO (Plain-old Java Object)
        \item Helper classes implemented with Rich-Domain Model and Data Mapper Java EE Design Patterns
        \item Typically one BO per table. Inheritence and Transaction Script design pattern (DAO) used when complexity ensues
      \end{itemize}
      \begin{s5notes}
        \begin{itemize}
          \item Business objects are used to help as a separation of concerns betwen granular functionality and abstract persistence layer.
          \item Does not have to be a POJO. Can contain functionality, but Kuali uses them strictly as POJO
          \item Talk about relationship to Domain Model
        \end{itemize}
        \begin{slideshow}
        \item \thumbnail{../Diagrams/PersistableBusinessObject_class.png}{../Diagrams/PersistableBusinessObject_class-thumb.png}
        \end{slideshow}
        \end{s5notes}
    \end{s5slide}

    \begin{s5slide}
      \section{OJB, Spring, and DAO's...again}
      \subsection{\xml{u}PersistenceBroker\xml{/u} (PB or JPB, but not PBJ, PB and J, or PB'n'J)}
      \begin{itemize}
      \item It's responsible for Spring/OJB/DAO integration. 
      \item Without \xml{u}PersistenceBroker\xml{/u}, DAO's are just plain services.
      \item \xml{u}AccountingLineDaoOjb\xml{/u}
      \item \xml{u}PlatformAwareDaoBaseOjb\xml{/u}
      \item \xml{u}PersistenceBrokerTemplate\xml{/u}
      \item \xml{u}PersistenceBrokerDaoSupport\xml{/u} 
      \end{itemize}
      
      \begin{slideshow}
      \item \thumbnail{../Diagrams/PersistenceBroker_class.png}{../Diagrams/PersistenceBroker_class-thumb.png}
      \end{slideshow}

      \begin{s5notes}
        Persistence broker integrates Service with OJB. Show diagrams and code.
      \end{s5notes}
    \end{s5slide}

\end{ifhtml}