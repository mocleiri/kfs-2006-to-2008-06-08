\begin{ifhtml}
    \begin{s5slide}
        \section{OJB and Business Objects}
        \begin{itemize}
          \item Domain Model Review
          \item OJB Mapping Review
          \item BusinessObjects
          \item OJB, Spring, and DAO's ... again
          \item \xml{u}PersistenceBroker\xml{/u} (PB or JPB, but not PBJ, PB and J, or PB'n'J)
            \begin{itemize}
              \item \xml{u}AccountingLineDao\xml{/u}
              \item \xml{u}PlatformAwareDaoBase\xml{/u}
              \item \xml{u}PersistenceBrokerTemplate\xml{/u}
              \item \xml{u}PersistenceBrokerDaoSupport\xml{/u} 
            \end{itemize}
        \end{itemize}
        \begin{s5notes}
        \end{s5notes}
    \end{s5slide}
    
    \begin{s5slide}
      \section{Domain Model Review}
      \begin{itemize}
        \item Two distinctions (Rich and Simple) Domain Models
        \item Domain Model represents an architecture where business objects and business rules are 
          abstracted from functionality and persistence layers
        \item Kuali uses Spring to create a service layer to act as interfaces to business objects and
          rules
        \item Kuali uses OJB to represent an abstraction of the persistence layer
        \item Kuali uses something called a \xml{u}PersistenceBroker\xml{/u} for the service layer to
          contact the persistence layer
      \end{itemize}
      \begin{s5notes}
        \begin{itemize}
        \item Two distinctions (Rich and Simple) Domain Models
        \item Domain Model represents an architecture where business objects and business rules are 
          abstracted from functionality and persistence layers
        \item Kuali uses Spring to create a service layer to act as interfaces to business objects and
          rules
        \item Kuali uses OJB to represent an abstraction of the persistence layer
        \item Kuali uses something called a \xml{u}PersistenceBroker\xml{/u} for the service layer to
          contact the persistence layer
        \end{itemize}
      \end{s5notes}
    \end{s5slide}

    \begin{s5slide}
      \section {OJB Mapping Review}
      \begin{itemize}
        \item Like Spring and JUnit, supports programming to interface.
        \item Mappings try to handle RDBMS table structure relationships and object-oriented hierarchy relationships
           at the same time. Tricky.
         \item In a Simple Domain Model, there mostly mappings for 1-to-1 table/java object.
      \end{itemize}
      
      \begin{s5notes}
        \begin{itemize}
          \item OJB allows interface mappings. This means that a table can be mapped to a Java interface.
          \item OJB can handle inheritence. That means more than one class/interface using the same table, 
            or multiple tables.
          \item Not advocating OJB, so much as the power of ORM's and Spring PersistenceBroker.
          \item Kuali uses the Simple Domain Model or Rich Domain Model?
        \end{itemize}
      \end{s5notes}
    \end{s5slide}

    \begin{s5slide}
      \section{Business Objects}
      \begin{s5notes}
        \end{s5notes}
    \end{s5slide}

    \begin{s5slide}
      \section{OJB, Spring, and DAO's...again}
      \begin{s5notes}
      \end{s5notes}
    \end{s5slide}

    \begin{s5slide}
      \section{\xml{u}PersistenceBroker\xml{/u}}
      \begin{s5notes}
      \end{s5notes}
    \end{s5slide}
\end{ifhtml}