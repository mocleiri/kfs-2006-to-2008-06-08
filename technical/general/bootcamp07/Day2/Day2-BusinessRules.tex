\begin{ifhtml}
    \begin{s5slide}
        \section{Events and Business Rules}
        \begin{itemize}
          \item How are events used in Kuali?
          \item How are events triggered?
          \item Different classes of events in KFS
          \item Scenarios for new events
          \item The \xml{u}KualiRulesService\xml{/u}
        \end{itemize}
        
        \begin{s5notes}
          
        \end{s5notes} 
    \end{s5slide}

    \begin{s5slide}
        \section{How are events used in Kuali?}
        \begin{itemize}
          \item These events are strictly concerning the Kuali Nervous System 
          \item They are the sentry for business rules execution.
          \item Events are open-ended 
            \begin{itemize}
            \item Events can occur for exceptions
            \item UI changes can result in event triggering
            \item Workflow routing status change can trigger an event
            \item (opposite of above) An event can trigger workflow routing of some kind
            \end{itemize}
        \end{itemize}
        
        \begin{s5notes}
          Events triggering is responsible for business rules execution. When an event is triggered, 
          the event framework handles the event and executes the appropriate rule for it. Even though events are triggered 
          on some workflow routing, that does not make them ``workflow routing events.'' These are also not events for Swing/AWT.
          
        \end{s5notes} 
    \end{s5slide}

    \begin{s5slide}
        \section{How are events triggered?}
        \begin{itemize}
          \item Events can be triggered in any class, but are always triggered in response to some user input
          \item Event object instantiation usually occurs within the Struts \xml{u}Action\xml{/u}.
        \end{itemize}
        
        \begin{s5notes}
          When a user clicks a ``save'' button or ``add'' button, an event may be triggered for that. Let's look at some events.
        \end{s5notes} 
    \end{s5slide}
\end{ifhtml}