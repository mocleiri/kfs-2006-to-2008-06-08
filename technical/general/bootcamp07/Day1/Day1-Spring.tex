\begin{ifhtml}
  \begin{s5slide}
    \section{Introduction to Spring}
    \begin{itemize}
      \item Spring is a JEE Framework
      \item A Kuali Service
      \item \code{SpringServiceLocator}
      \item DAO's and Spring
      \item Unit Testing Services
    \end{itemize}

    \begin{s5notes}
      Thing's to go over
      \begin{itemize}
      \item Spring is a JEE Framework
      \item \code{SpringServiceLocator}
      \item DAO's and Spring
      \item Unit testing and Services        
      \end{itemize}
    \end{s5notes} 
  \end{s5slide}

  \begin{s5slide}
    \section{Spring is a JEE Framework}
    Some time back, EJB (Entity Beans in particular) had a monopoly on JEE. EJB was synonymous with JEE. EJB's had their share of 
    problems, and Spring addresses them in these ways.
    \begin{itemize}
      \item Spring is small footprint
      \item Abstract Transaction Management Layer
      \item Spring is an Inversion of Control Container
      \item Dependency Injection
    \end{itemize}

    \begin{s5notes}
      \begin{itemize}
        \item EJB is frequently regarded as being heavy. Heavy in implementation, and in development turnaround. It has been necessary
          even for small applications because of the Transaction Management. Spring gives transaction management for very little cost, 
          so it becomes attractive for simpler architectures. Sure Kuali is a really huge app, but EJB still isn't right for it because
          of the simple Domain Model architecture that Kuali is following. It is still overkill for EJB. There are other reasons that 
          EJB is unattractive for Kuali, but we won't go into that.
        \item EJB is very very difficult to test and debug because of the nature of RMI, and the fact that EJB is programmed to classes
          instead of interfaces. Spring makes it much easier to program to interface through Inversion of Control.
        \item Spring claims inversion of control by configuration of application components and lifecycle management of Java objects.
          Really, what this means is it defines an abstraction for relating Java Objects to frameworks and libraries and manages all of
          that for us.
        \item Spring is also really easy to configure because of introspection and Dependency Injection. Kuali uses XML to configure 
          services for Spring, but Spring does all of the work of introspection (determine an object's type at runtime,) and dependency 
          discovery/validation and aggregate assignment. That's what it means by injection. That's all part of being an IoC container.
        \item Dependency Injection is good because it's more orthogonal and modular type of development. Services can freely be 
          developed without having to worry about requisites and dependencies.
      \end{itemize}
    \end{s5notes} 
  \end{s5slide}

  \begin{s5slide}
    \section{A Kuali Service}
    The \code{BusinessObjectService} is used to handle storing and retrieving Business Objects. As a service, it is agnostic to persistence,
    but it does deal with persistent storage through something called the BusinessObjectDao.

    \begin{s5notes}
      \begin{itemize}
        \item Show a simple example of a Kuali Spring Service. BusinessObjectService (show config, interface, and impl)
      \end{itemize}
    \end{s5notes} 
  \end{s5slide}

  \begin{s5slide}
    \section{\code{SpringServiceLocator}}
    ``We have Dependency Injection. Why do we need something like \code{SpringServiceLocator}?'' 

    Our \MVC \ does not know anything about Spring, we have to contact Spring through a singleton. 

    \begin{s5notes}
      \begin{itemize}
        \item ``We have Dependency Injection. Why do we need something like \code{SpringServiceLocator}?'' Our \MVC \ does not know 
          anything about Spring, we have to contact Spring through a singleton. In most cases, Dependency Injection is used. When 
          it can't be used, we have \code{SpringServiceLocator}.
        \item Explore class diagram of relationships of the SpringServiceLocator
        \item Show sequence of service discovery through the SpringServiceLocator
      \end{itemize}
    \end{s5notes} 
  \end{s5slide}

  \begin{s5slide}
    \section{DAO's and Spring}
    \begin{itemize}
      \item DAO is (Data Access Object)
      \item Will discuss DAO's in more detail on Day 2
      \item DAO's are defined as ``beans'' or ``services'' in Spring configuration
      \item DAO's are injected for use into other services (Example, DocumentDao and LookupDao.)
    \end{itemize}

    \begin{s5notes}
      \begin{itemize}
      \item DAO is (Data Access Object)
      \item Will discuss DAO's in more detail on Day 2
      \item DAO's are defined as ``beans'' or ``services'' in Spring configuration. This makes them services pretty much.
      \item DAO's are injected for use into other services (Example, DocumentDao and LookupDao.)
      \item It is recommended not to write unit tests for DAO's, but rather for services they are injected into
      \end{itemize}
    \end{s5notes} 
  \end{s5slide}

  \begin{s5slide}
    \section{Unit Testing Services}
    \begin{itemize}
      \item KFS Unit Testing is of Services mostly rather than Business Objects or DAO's
      \item Kuali and TDD
        \begin{itemize}
          \item Test first
          \item red-green pattern
          \item Unit Tests vs. Integration Tests
        \end{itemize}
      \item KFS has a special unit testing extension framework with annotations for testing Spring Services
    \end{itemize}

    \begin{s5notes}
      \begin{itemize}
      \item KFS Unit Testing is of Services mostly rather than Business Objects or DAO's, so it makes sense to talk about unit 
        testing and services at the same time.
      \item TDD is not a hard-and-fast requirement, it is hugely beneficial though. Most people that use it, can't imagine not 
        using it. The basic idea is to write tests for interfaces (no implementation yet.) This gets tests to run and compile. 
        At first, tests always fail. As you develop implementations, the tests start to pass. This is referred to as ``red-green'' 
        because you start red and become green later.
      \item Look at examples of services unit tests. (BusinessObjectService unit tests.)
      \item Discuss unit tests are what we call the tests we write. Code Coverage is important. 
      \end{itemize}
    \end{s5notes} 
  \end{s5slide}

  \begin{s5slide}
    \section{Spring Workshop Session}
    \subsection{Goals}
    \begin{enumerate}
      \item Create and register a service that interacts with the Mock Persistance DAO
      \item Integrate the Service into our Struts Action from the Struts Workshop
      \item Build and run unit tests for the service.
    \end{enumerate}

  \end{s5slide}
\end{ifhtml}