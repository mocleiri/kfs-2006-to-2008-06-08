\begin{ifhtml}
    \begin{s5slide}
        \section{What is DWR?}
            \begin{itemize}
                \item Direct Web Remoting (DWR) is:
                \begin{itemize}
                    \item An Ajax framework - allows you to abstract your Ajax needs beyond just an XHR request
                    \item Java-focused - makes it easier to configure your services and beans to be called through Ajax
                    \item Simple and easy to install and use
                \end{itemize}
            \end{itemize}
    \end{s5slide}
    \begin{s5slide}
        \section{What is Ajax?}
            \begin{itemize}
                \item Originally stood for Asynchronous JavaScript and XML
                \item Now more of a generic term no longer an acronym
                \begin{itemize}
                    \item Typically encompasses the following:
                    \begin{itemize}
                        \item XmlHttpRequest (XHR) calls to Web servers 
                        \item Doing something dynamic with those calls on the client
                        \item Client-side validation (with or without XHR, but most likely with)
                        \item Dynamic JavaScript effects (most likely combined with XHR calls)
                    \end{itemize}
                    \item Lumped in with the term "Web 2.0"
                \end{itemize}
            \end{itemize}
            \begin{s5notes}
                \begin{itemize}
                    \item Originally stood for Asynchronous JavaScript and XML
                    \item Now more of a generic term no longer an acronym
                    \begin{itemize}
                        \item Typically encompasses the following:
                        \begin{itemize}
                            \item XmlHttpRequest (XHR) calls to Web servers 
                            \item Doing something dynamic with those calls on the client
                            \item Client-side validation (with or without XHR, but most likely with)
                            \item Dynamic JavaScript effects (most likely combined with XHR calls)
                        \end{itemize}
                        \item Lumped in with the term "Web 2.0"
                    \end{itemize}
                \end{itemize}
            \end{s5notes}
    \end{s5slide}
    \begin{s5slide}
        \section{Ajax Layers}
            \begin{itemize}
                \item Low level – XMLHttpRequest, IFrame (used to store and pass information w/out XMLHttpRequest)
                \item Remoting Toolkit – DWR, JSON-RPC, dojo.io.bind(), Prototype
                \item UI Toolkit – Dojo, SmartClient, Backbase, etc., Script.aculo.us
                \item Ajaxian Web Frameworks – Tapestry, Rails, ASP.NET
            \end{itemize}
            \begin{itemize}
                \item DWR Currently sites in-between Remoting and UI Toolkit and moving more into the UI Toolkit realm
            \end{itemize}
            \begin{s5notes}
                \begin{itemize}
                    \item Low level – XHR/iframe
                    \item Remoting Toolkit
                    \item UI Toolkit - Dojo/Scriptaculous
                    \item Ajaxian Web Frameworks – Tapestry, Rails, ASP.NET
                \end{itemize}
                \begin{itemize}
                    \item DWR Currently sites in-between Remoting and UI Toolkit and moving more into the UI Toolkit realm
                \end{itemize}
            \end{s5notes}
    \end{s5slide}
    \begin{s5slide}
        \section{Ajax Adoption rates}
            \begin{itemize}
                \item Denial
                \item Progressive Enhancement
                \item Second Site
                \item Accessible JavaScript
            \end{itemize}
            \begin{s5notes}
                \begin{itemize}
                    \item Denial
                    \item Progressive Enhancement
                    \item Second Site
                    \item Accessible JavaScript
                \end{itemize}
                \begin{itemize}
                    \item Currently KFS is in "Progressive Enhancement" - we are slowly adding new pieces here and there, but it isn't a priority for KFS
                    \item KRA though (and Leo can speak to this) will be more Ajaxy
                \end{itemize}
            \end{s5notes}
    \end{s5slide}
    \begin{s5slide}
        \section{DWR and Java}
            \begin{itemize}
                \item DWR is designed to be used specifically with Java (the server-side piece of it)   
                \item But this doesn't preclude you in using DWR's client-side piece on non-Java sites
            \end{itemize}
            \begin{itemize}
                \item Some neat features of DWR:
                \begin{itemize}
                    \item Can automatically generate the necessary JavaScript on the fly based on your configuration
                    \item Extremely easy to control what access there is to your beans and services
                    \item Easy to configure bean creation
                \end{itemize}
            \end{itemize}
            
        \begin{s5notes}
            \begin{itemize}
                \item DWR/Java link   
                \item but can still use client-side with anything you want
            \end{itemize}
            \begin{itemize}
                \item Neat features:
                \begin{itemize}
                    \item generates JavaScript on the fly
                    \item Access control of beans/services
                    \item Easy to configure bean creation
                \end{itemize}
                \item All of these we will get to shortly
            \end{itemize}
        \end{s5notes}
    \end{s5slide}
    \begin{s5slide}
        \section{How do we use DWR in KFS?}
            \begin{itemize}
                \item Two easy to spot places:
                \begin{itemize}
                    \item User lookups
                    \item Chart/Account lookups
                \end{itemize}
            \end{itemize}
        \begin{s5notes}
            Show a username being filled in, show an account number being filled in within a running Kuali
        \end{s5notes}
    \end{s5slide}
    \begin{s5slide}
        \section{What is involved with \\getting this to work?}
            \begin{itemize}
                \item First the actual DWR configuration
                \item Then we have the generated JavaScript
                \item Then we wrap that JavaScript in our own JavaScript library so we pass in the right info to it
            \end{itemize}
        \begin{s5notes}
            \begin{itemize}
                \item Show the DWR configuration file (dwr-core.xml, dwr-chart.xml)
                \item Show the generated JavaScript for account and user stuff (AccountService, UserService)
                \item Show our JavaScript that uses the account and user JavaScript (accountGlobal.js, objectInfo.js)
                \item \code{param name="include"} - what parameters to 
                
            \end{itemize}
        \end{s5notes}
    \end{s5slide}
    \begin{s5slide}
        \section{Behind the scenes Spring bean creation}
            \begin{itemize}
                \item We have created our own \code{SpringCreator} called \code{GlobalResourceDelegatingSpringCreator} that handles our bean creation for us
                \item Connects to our Spring context and attempts to retrieve a bean instance for us
                \item Primarily customized to report back more helpful error messages
            \end{itemize}
            \begin{s5notes}
                \begin{itemize}
                    \item Show our GlobalResourceDelegatingSpringCreator (accountGlobal.js, objectInfo.js)
                \end{itemize}
            \end{s5notes}
    \end{s5slide}
    \begin{s5slide}
        \section{Debugging DWR instance}
            \begin{itemize}
                \item Easy to configure (but make sure you turn it off before deploying publicly)
                \item In our servlet context we can set a debug parameter
            \end{itemize}
            \begin{s5notes}
                \begin{itemize}
                    \item Show changing dwr-invoker to true and restarting app to see all the services available
                    \item Demo actually looking up an account number through the web interface to test DWR
                \end{itemize}
            \end{s5notes}
    \end{s5slide}
\end{ifhtml}