\documentclass[12pt,notitlepage]{article}
\author{Leo Przybylski \\
\texttt{przybylskil@arizona.edu}}
\title{Inheritance vs. Encapsulation}

\begin{document}
  \maketitle
  \begin{abstract}
    There are some instances in \emph{Kuali Financial System} where Inheritance is possibly misused where Encapsulation may have not
    only been a better fit, but may have also lessened or even eliminated the need for refactoring. This document explains how Encapsulation
    can mitigate the need for future refactorings. Inheritance as it is now in Java limits modularity and orthogonal design.
  \end{abstract}
  
  \section{\hfill Introduction}
  \hrulefill
  \\
  \begin{quote}
  For all things object-oriented, the conceptual framework is the \emph{object model}. --Grady Booch ~\cite{gbone}
  \end{quote}
  
  The object model Grady Booch refers to is made up of \emph{Abstraction}, \emph{Encapsulation}, \emph{Modularity}, and \emph{Hierarchy}.
  People have different ideas about what inheritance is. It is important to establish such definitions to have a clear understanding
  of what is being discussed. For this document, I use the following definitions:
  \begin{description}
    \item[Inheritance] is if and only if the \texttt{extends} directive is used with the \texttt{class} or \texttt{interface} clause. 
      It is important to understand that Java does not support multiple-inheritance even though it does support polymorphism. Implementing 
      an interface does not constitute inheritance. 

      See the following example:
      \begin{verbatim}
public abstract class AbstractGLLookupableHelperServiceImpl 
                extends AbstractLookupableHelperServiceImpl {
    // ...
}
      \end{verbatim}

      \begin{quote}
        Semantically, inheritance denotes an ``is-a'' relationship. \ldots 
        Inheritance thus implies a genralization/specialization hierarchy, 
        wherein a subclass specializes the more general structure or behavior of its superclasses. --Grady Booch ~\cite{gbtwo}
      \end{quote}
      
      Based on the above quote from Grady Booch, we can assume that inheritance forms a kind of hierarchy which is an element in the 
      \emph{object model}.

      Java does not support multiple-inheritance. The only hierarchy choice is for single-inheritance through the \texttt{extends} clause as
      described above. Simply put:
      \begin{description}
      \item[Single-Inheritance] is when one class inherits from a single superclass. The visual representation of such a hierarchy always looks
        like a tree.
      \end{description}

    \item[Encapsulation] is a technique of enclosing class functionality through object instances within another class. Many 
      design patterns employ encapsulation. Among these patterns are Delegate/Aggregate, Composite, Decorator, and Adapter. Some may know these
      patterns by other names. Encapsulation can be by implementation (as in implementing and abstraction,) or aggregation.  Grady Booch 
      calls aggregation a ``part of'' relationship.
  \end{description}

  In simple terms, inheritance and encapsulation are both intended for implementation. Inheritance can be called an ``is a'' relationship, and
  encapsulation can be called a ``part of'' relationship type. Inheritance and encapsulation are both complementary to abstraction.

  \subsection{What is the Problem?}
  The problem is that Java supports polymorphism, but not multiple-inheritance. As a result due to the lack of multiple-inheritance, Java 
  developers have designed object-oriented patterns to sidestep single-polymorphism. Single-inheritance creates a single hierarchy that 
  resembles a tree where there is only one class at the very top. Eg., for \sf Java\rm, the \texttt{Object} class is the top-most class in the hierarchy.
  
  My Points:
  \begin{itemize}
    \item Java only has single-inheritance
    \item In order to achieve polymorphism and somewhat multi-inheritance, Java uses \texttt{interfaces}.
    \item Interfaces are abstractions
    \item Java has a confusing \texttt{abstract} class
  \end{itemize}
  \subsection{Topics}
  \begin{itemize}
    \item When to use Inheritance?
    \item When to use Encapsulation?
    \item What are examples of Encapsulation in Kuali
    \item What are patterns of bad uses of Inheritance?
    \item Why?
  \end{itemize}

  \section{\hfill Inheritance is Part of OOP. Doesn't that Make it a Good Thing?}
  \hrulefill

  \section{\hfill Encapsulation is Also Part of OOP, but Why is it Better than Inheritance?}
  \hrulefill
  \begin{thebibliography}{20}
    \bibitem{gbone} ~Object-Oriented Analysis and Design, Grady Booch, (2000) 
    \bibitem{gbtwo} ~Object-Oriented Analysis and Design, Grady Booch, (2000) 
  \end{thebibliography}
\end{document}