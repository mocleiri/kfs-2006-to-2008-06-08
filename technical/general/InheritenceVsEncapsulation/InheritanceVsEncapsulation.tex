\documentclass[12pt,notitlepage]{article}
\author{Leo Przybylski \\
\texttt{przybyls@u.arizona.edu}}

\usepackage{listings}
\usepackage{color}
\usepackage{hyperref}
\usepackage{hyperlatex}
\setcounter{htmldepth}{0}

\definecolor{DarkBlue}{rgb}{0,0,0.55}
\definecolor{DarkGreen}{rgb}{0,0.4,0}
\definecolor{Purple}{rgb}{0.5,0,0.5}

\begin{ifhtml}
\newcommand{\sf}[1]{\xml{span style="font-family: sans-serif;"}#1\xml{/span}}
\newcommand{\rm}[1]{\xml{span style="font-family: serif;"}#1\xml{/span}}
\newcommand{\bf}[1]{\xml{span style="font-weight: bold;"}#1\xml{/span}}
\newcommand{\href}[2]{\xml{a href="#1"}#2\xml{/a}}
\newcommand{\HlxStyleSheet}{
  \begin{rawxml}
    <!-- metadata -->
    <meta name="generator" content="S5" />
    <meta name="version" content="S5 1.1" />
    <meta name="presdate" content="\end{rawxml}\HlxDate\begin{rawxml}"/>
    <meta name="author" content="\end{rawxml}\HlxAuthor\begin{rawxml}"/>
    <meta name="company" content="Leosandbox " />
    <!-- configuration parameters -->
    <meta name="defaultView" content="slideshow" />
    <meta name="controlVis" content="hidden" />
    <!-- style sheet links -->
    <link rel="stylesheet" href="ui/kuali/slides.css" type="text/css" media="projection" id="slideProj" />
    <link rel="stylesheet" href="ui/kuali/outline.css" type="text/css" media="screen" id="outlineStyle" />
    <link rel="stylesheet" href="ui/kuali/print.css" type="text/css" media="print" id="slidePrint" />
    <link rel="stylesheet" href="ui/kuali/opera.css" type="text/css" media="projection" id="operaFix" />
    <script src="ui/kuali/slides.js" type="text/javascript"></script>
  \end{rawxml}
}

\newcommand{\maketitle}{
    \xml{div class="slide"}
    \EmptyP{\HlxTitleP}{\HlxBlk\xml{h1}\HlxTitle\xml{/h1}
    \EmptyP{\HlxAuthorP}{\xml{h2}\HlxAuthor\xml{/h2}}{}
    \EmptyP{\HlxDate}{\xml{h3}\HlxDate\xml{/h3}}{}
    \xml{/div}
    }{}}

\end{ifhtml}

\newenvironment{s5presentation}{\endpar%
  \HlxBlk\begin{rawxml}
  <div class="header">
    <div class="header_l">
      <div class="header_r">
      &nbsp;
      </div>
    </div>
  </div>
  <div class="content">
    <div class="content_l">
      <div class="content_r">
        <img src="ui/kuali/blank.gif" id="filler" border="0"/>
<div class="layout">
  <div id="controls"><!-- DO NOT EDIT --></div>
  <div id="currentSlide"><!-- DO NOT EDIT --></div>
  <div id="header">
    <img style="position: relative; top:30px; left: 18px;" src="ui/kuali/logo.png" />
  </div>
  <div id="footer">
    <div class="footer_l">
      <div class="footer_r">
        <h1/>
        <h2>\end{rawxml}\HlxTitle\begin{rawxml}</h2>
      </div>
    </div>
  </div>
</div>
    </div>
  </div>
 </div>
<div class="presentation">
    \end{rawxml}}{\HlxBlk\xml{/div}}
\newenvironment{s5slide}{\xml{div class="slide"}}{\xml{/div}}
\newcommand{\fulltitle}[1]{\title{#1}\htmlonly{\htmltitle{#1}}}

\fulltitle{Inheritance vs. Encapsulation}

\newcommand{\Encapsulation}{\sf{\href{http://en.wikipedia.org/wiki/Information_hiding}{Encapsulation}}}
\newcommand{\Abstraction}{\sf{\href{http://en.wikipedia.org/wiki/Abstraction}{Abstraction}}}
\newcommand{\Inheritance}{\sf{\href{http://en.wikipedia.org/wiki/Inheritance_(computer_science)}{Inheritance}}}
\newcommand{\ObjectModel}{\sf{\href{http://en.wikipedia.org/wiki/Object_model}{Object Model}}}
\newcommand{\Modularity}{\sf{\href{http://en.wikipedia.org/wiki/Modularity_(programming)}{Modularity}}}
\newcommand{\Polymorphism}{\sf{\href{http://en.wikipedia.org/wiki/Type_polymorphism}{Polymorphism}}}
\newcommand{\MultiInheritance}{\sf{\href{http://en.wikipedia.org/wiki/Multiple_inheritance}{Multiple-Inheritance}}}
\newcommand{\Kuali}{\emph{\href{http://www.kuali.org}{Kuali}}}
\newcommand{\KualiFinancialSystem}{\emph{\href{http://www.kuali.org}{Kuali Financial System}}}
\newcommand{\KFS}{\emph{\href{http://www.kuali.org}{KFS}}}

\begin{document}
  \texonly{
    \lstset{language=Java,
      basicstyle=\small,
      keywordstyle=\color{Purple}\bfseries,
      commentstyle=\color{DarkGreen},
      identifierstyle=\color{DarkBlue},
      rangeprefix=\/\/\ ,
      rangesuffix=,
      breaklines=true,
      breakatwhitespace=true,
      includerangemarker=false}
  }
  \W \begin{s5presentation}
  \maketitle
  \W \begin{s5slide}
    \texonly{\tableofcontents}
  \W \end{s5slide}
  \W \begin{s5slide}
    \begin{abstract}
      \begin{itemize}
        \item Some may already know a lot of this (it's for context.)
        \item Try to quickly move passed boring material, but without ignoring context.
        \item Do not be distracted by the fancy drawings.
        \item Leo's $10^{th}$ anniversary is March 27, 2007
        \item \href{http://en.wikipedia.org/wiki/Education_and_Sharing_day}{March 27 is...}
      \end{itemize}
    \end{abstract}
    \W \end{s5slide}
  
  \W \begin{s5slide}
    \section{\hfill Introduction}
    \T \hrulefill
    \T \

    There are some instances in \KualiFinancialSystem\ where \Inheritance\ is possibly misused where \Encapsulation\ may have not
    only been a better fit, but may have also lessened or even eliminated the need for refactoring. This document explains how 
    \Encapsulation\ can mitigate the need for future refactorings. \Inheritance\ as it is now in Java limits modularity and orthogonal design.
    \W \end{s5slide}

    \begin{ifhtml}
      \begin{s5slide}
        \section{\hfill Introduction}
      \begin{itemize}
      \item The \ObjectModel
        \begin{itemize}
          \item Origins of \Inheritance\ and \Encapsulation
          \item \Inheritance\ and \Encapsulation\ Defined
        \end{itemize}
      \item What is the Problem?
      \item Comparing \Inheritance\ and \Encapsulation\
        \begin{itemize}
          \item How do They Differ?
          \item Labor Distribution Scenario
          \item General Ledger Scenario
          \item Struts/Spring Scenario (aka. Encapsulating Actions or Does Spring + Struts = String?)
          \item Solving Parallel Hierarchies
        \end{itemize}
      \end{itemize}
      \end{s5slide}
    \end{ifhtml}

    \W \begin{s5slide}
      \W \section{The Object Model}
      \begin{quote}
        For all things object-oriented, the conceptual framework is the \ObjectModel. --Grady Booch ~\cite{gbone}
      \end{quote}
      
    The object model Grady Booch refers to is made up of \Abstraction, \Encapsulation, \Modularity\, and \emph{Hierarchy}.
    People have different ideas about what \Inheritance\ is. 
    \W \end{s5slide}
  
  \W \begin{s5slide}
    \texorhtml{\subsection{\Inheritance\ and \Encapsulation Defined}}
    {\section{Inheritance and\\ Encapsulation Defined}}
    It is important to establish such definitions to have a clear understanding
    of what is being discussed. For this document, I use the following definitions:
    \begin{ifhtml}
    \begin{itemize}
    \item What Does "\Inheritance" Mean for this Presentation?
    \item What Does "\Encapsulation" Mean for this Presentation?
    \end{itemize}
    \end{ifhtml}
  \W \end{s5slide}
  
  \begin{description}
    \W \begin{s5slide}
      \W \section{Inheritance is\ldots}
    \item[\Inheritance] is if and only if the \texttt{extends} directive is used with the \texttt{class} or \texttt{interface} clause. 
      It is important to understand that Java does not support \MultiInheritance. Java can achieve \Polymorphism by a \texttt{class} 
      implementing an \texttt{interface}.
      
      See the following example:
      \W \end{s5slide}
    
    \begin{ifhtml}
      \begin{s5slide}
        \section{An Inheritance Example}
        \begin{figure}
          \caption{An Inheritance Example in UML}
        \begin{verbatim}
public abstract class AbstractGLLookupableImpl 
    extends KualiLookupableImpl {
    ...
    ...
}        
        \end{verbatim}
        
        \center{\htmlimg{Diagrams/InheritanceExample_class.png}{Inheritance Example}}
        \end{figure}
      \end{s5slide}
    \end{ifhtml}
      
    \begin{tex}
      \lstinputlisting[linerange=ClassSignatureStart-ClassSignatureEnd,
                        caption={\Inheritance\ Definition Example}]{src/AbstractGLLookupableImpl.java}
      \begin{lstlisting}
    ...
    ...
}
      \end{lstlisting}

      \begin{quote}
        Semantically, \Inheritance\ denotes an ``is-a'' relationship. \ldots 
        \Inheritance\ thus implies a genralization/specialization hierarchy, 
        wherein a subclass specializes the more general structure or behavior of its superclasses. --Grady Booch ~\cite{gbtwo}
      \end{quote}
      Based on the above quote from Grady Booch, we can assume that \Inheritance\ forms a kind of hierarchy which is an element in the 
      \ObjectModel.

      Java does not support \MultiInheritance. The only hierarchy choice is for single-inheritance through the \texttt{extends} clause as
      described above. Simply put:
      \begin{description}
      \item[Single-\Inheritance] is when one class inherits from a single superclass. The visual representation of such a hierarchy always looks
        like a tree.
      \end{description}
      \end{tex}

      \W \begin{s5slide}
        \W \section{Encapsulation is...}
        \item[\Encapsulation] is a technique of enclosing class functionality through object instances within another class. Many 
          design patterns employ \Encapsulation. Among these patterns are Delegate/Aggregate, Composite, Decorator, and Adapter (some may know these
          patterns by other names.) \Encapsulation\ can be by implementation (as in implementing and abstraction,) or aggregation.  Grady Booch 
          calls aggregation a ``part of'' relationship. See the following example:
          \W \end{s5slide}
      
      \begin{ifhtml}
        \begin{s5slide}
          \section{An Encapsulation Example}
          \href{http://en.wikipedia.org/wiki/Dependency_Injection}{Dependency Injection} is a very widely-used example of \Encapsulation\
          throughout \KFS. Basically, \href{http://en.wikipedia.org/wiki/Dependency_Injection}{Dependency Injection} achieves \emph{Loose Coupling}
          or \Modularity\ by  encapsulating functionality from other classes through delegating the implementations.
        \end{s5slide}
      \end{ifhtml}
  \end{description}

  
  \begin{tex}
  In simple terms, inheritance and encapsulation are both intended for implementation. Inheritance can be called an ``is a'' relationship, and
  encapsulation can be called a ``part of'' relationship type. Inheritance and encapsulation are both complementary to abstraction.
  \end{tex}

  \T \subsection{What is the Problem?}
  \W \begin{s5slide}
    \W \section{What is the Problem?}
    The problem is that Java supports \Polymorphism, but not \MultiInheritance. 
    Single-\Inheritance\ creates a single hierarchy that resembles a tree where there 
    is only one class at the very top (eg., for \sf{Java}, the \texttt{Object} class is the top-most class in the hierarchy.)
    Resulting from the lack of \MultiInheritance, Java developers have designed object-oriented 
    patterns to sidestep single-\Polymorphism\ using \Encapsulation\ rather than \Inheritance.
    \W \end{s5slide}

  \begin{tex}
  My Points:
  \begin{itemize}
    \item Java only has single-inheritance
    \item In order to achieve \Polymorphism\ and somewhat \MultiInheritance, Java uses \texttt{interfaces}.
    \item Interfaces are abstractions
    \item Java has a confusing \texttt{abstract} class
  \end{itemize}
  \subsection{Topics}
  \begin{itemize}
    \item When to use \Inheritance?
    \item When to use \Encapsulation?
    \item What are examples of \Encapsulation\ in Kuali
    \item What are patterns of bad uses of \Inheritance?
    \item Why?
  \end{itemize}

  \section{\hfill Inheritance is Part of OOP. Doesn't that Make it a Good Thing?}
  \texorhtml{\hrulefill}{\htmlrule}

  \section{\hfill Encapsulation is Also Part of OOP, but Why is it Better than Inheritance?}
  \end{tex}
  \T \hrulefill

  \begin{ifhtml}
    \begin{s5slide}
      \section{How Do They Differ?}
      \begin{itemize}
        \item They are different kinds of relationships
          \begin{itemize}
          \item \Inheritance\ is an "is a" relationship type
          \item \Encapsulation\ is a "part of" relationship type
          \end{itemize}
        \item \Modularity\ and Coupling
          \begin{itemize}
            \item \Inheritance\ leads to tight coupling
            \item \Encapsulation\ leads to loose coupling
          \end{itemize}
        \item \Polymorphism
      \end{itemize}
    \end{s5slide}
  \end{ifhtml}
  
  \begin{ifhtml}
    \begin{s5slide}
      \section{Labor Distribution Scenario}
    \end{s5slide}
  \end{ifhtml}

  \begin{ifhtml}
    \begin{s5slide}
      \section{General Ledger Scenario}
      \begin{figure}
        \center{\htmlimg{Diagrams/GlLookupableImpl_class.png}{General Ledger Account Balance Lookupable}}
      \end{figure}
    \end{s5slide}
  \end{ifhtml}

  \W \end{s5presentation}
  \begin{tex}
    \begin{thebibliography}{20}
    \bibitem{gbone} ~Object-Oriented Analysis and Design, Grady Booch, (2000) 
    \bibitem{gbtwo} ~Object-Oriented Analysis and Design, Grady Booch, (2000) 
    \end{thebibliography}
  \end{tex}
\end{document}