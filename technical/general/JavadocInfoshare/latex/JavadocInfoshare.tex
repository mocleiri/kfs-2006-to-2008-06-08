\documentclass[12pt,notitlepage]{article}
\author{Leo Przybylski \\
\texttt{przybyls@u.arizona.edu}}
\usepackage{graphicx}
\usepackage{listings}
\usepackage{color}
\usepackage{hyperref}
\usepackage{hyperlatex}
\setcounter{htmldepth}{0}

\definecolor{DarkBlue}{rgb}{0,0,0.55}
\definecolor{DarkGreen}{rgb}{0,0.4,0}
\definecolor{Purple}{rgb}{0.5,0,0.5}

\begin{ifhtml}
\newcommand{\sf}[1]{\xml{span style="font-family: sans-serif;"}#1\xml{/span}}
\newcommand{\rm}[1]{\xml{span style="font-family: serif;"}#1\xml{/span}}
\newcommand{\bf}[1]{\xml{span style="font-weight: bold;"}#1\xml{/span}}
\newcommand{\href}[2]{\xml{a href="#1"}#2\xml{/a}}
\newcommand{\HlxStyleSheet}{
  \begin{rawxml}
    <!-- metadata -->
    <meta name="generator" content="S5" />
    <meta name="version" content="S5 1.1" />
    <meta name="presdate" content="\end{rawxml}\HlxDate\begin{rawxml}"/>
    <meta name="author" content="\end{rawxml}\HlxAuthor\begin{rawxml}"/>
    <meta name="company" content="Leosandbox " />
    <!-- configuration parameters -->
    <meta name="defaultView" content="slideshow" />
    <meta name="controlVis" content="hidden" />
    <!-- style sheet links -->
    <link rel="stylesheet" href="ui/kuali/slides.css" type="text/css" media="projection" id="slideProj" />
    <link rel="stylesheet" href="ui/kuali/outline.css" type="text/css" media="screen" id="outlineStyle" />
    <link rel="stylesheet" href="ui/kuali/print.css" type="text/css" media="print" id="slidePrint" />
    <link rel="stylesheet" href="ui/kuali/opera.css" type="text/css" media="projection" id="operaFix" />
    <script src="ui/kuali/slides.js" type="text/javascript"></script>
  \end{rawxml}
}

\newcommand{\maketitle}{
    \xml{div class="slide"}
    \EmptyP{\HlxTitleP}{\HlxBlk\xml{h1}\HlxTitle\xml{/h1}
    \EmptyP{\HlxAuthorP}{\xml{h2}\HlxAuthor\xml{/h2}}{}
    \EmptyP{\HlxDate}{\xml{h3}\HlxDate\xml{/h3}}{}
    \xml{/div}
    }{}}

\end{ifhtml}

\newenvironment{s5presentation}{\endpar%
  \HlxBlk\begin{rawxml}
  <div class="header">
    <div class="header_l">
      <div class="header_r">
      &nbsp;
      </div>
    </div>
  </div>
  <div class="content">
    <div class="content_l">
      <div class="content_r">
        <img src="ui/kuali/blank.gif" id="filler" border="0"/>
<div class="layout">
  <div id="controls"><!-- DO NOT EDIT --></div>
  <div id="currentSlide"><!-- DO NOT EDIT --></div>
  <div id="header">
    <img style="position: relative; top:30px; left: 18px;" src="ui/kuali/logo.png" />
  </div>
  <div id="footer">
    <div class="footer_l">
      <div class="footer_r">
        <h1/>
        <h2>\end{rawxml}\HlxTitle\begin{rawxml}</h2>
      </div>
    </div>
  </div>
</div>
    </div>
  </div>
 </div>
<div class="presentation">
    \end{rawxml}}{\HlxBlk\xml{/div}}
\newenvironment{s5slide}{\xml{div class="slide"}}{\xml{/div}}
\newcommand{\fulltitle}[1]{\title{#1}\htmlonly{\htmltitle{#1}}}

\fulltitle{Javadoc Infoshare}

\newcommand{\Kuali}{\emph{\href{http://www.kuali.org}{Kuali}}\rm{}}
\newcommand{\KualiFinancialSystem}{\emph{\href{http://www.kuali.org}{Kuali Financial System}}\rm{}}
\newcommand{\KFS}{\emph{\href{http://www.kuali.org}{KFS}}}

\begin{document}
  \texonly{
    \lstset{language=Java,
      basicstyle=\small,
      keywordstyle=\color{Purple}\bfseries,
      commentstyle=\color{DarkGreen},
      identifierstyle=\color{DarkBlue},
      rangeprefix=\/\/\ ,
      rangesuffix=,
      breaklines=true,
      breakatwhitespace=true,
      includerangemarker=false}
  }
  \W \begin{s5presentation}
    \maketitle
    \begin{ifhtml}
      \begin{s5slide}
        \section{Goals}
        \begin{itemize}
          \item Show how to merge other forms of documentation into Javadocs
          \item Demonstrate automated documentation possibilities
          \item Demonstrate other methods for merging code and documentation
          \item Confluence is currently a private store of documentation. Howtos, Guides, API 
            documentation, etc... can be a publicly released distribution. Let this Infoshare
            inspire possibilities for publicly disseminated documentation.
        \end{itemize}
      \end{s5slide}

      \begin{s5slide}
        \section{Things to Go Over}
        \begin{enumerate}
          \item Review Directory Structure
          \item Javadoc overview.html and package.html Files
          \item Merging Code with \LaTeX
        \end{enumerate}
      \end{s5slide}

      \begin{s5slide}
        \section{Directory Structure}
        \begin{itemize}
      \item \textbf{Diagrams}
        ER or UML diagrams are stored here. Can be SVG
      \item \textbf{latex}
        The \LaTeX document standard format files are stored here. Not exactly proprietary, but
        it's not MS Word either. LaTeX files have to be compiled into DVI, then converted to PDF for
        the build.
      \item \textbf{resources}
          Graphics, documents in PDF, RTF, Word, etc... can be stored here. Files here become
          available in the resources directory of the documentation build.
        \item \textbf{src}
          The distribution is indenpendent of the documentation; nevertheless, the 
          documentation still requires the distribution to build javadocs. This directory 
          is not necessary if there is an available distribution to point to.
      \end{itemize}
      \end{s5slide}
    \begin{s5slide}
      \section{Javadoc overview.html and package.html Files}
      Diagrams, Howto's, Guides, and other documentation describing details about usage and development 
      can be embedded in the package.html and overview.html files.
    \end{s5slide}
    \begin{s5slide}
      \section{Merging Code with \LaTeX}
      \LaTeX is pronounced (Lah-Tek) or (Lay-Tek). It is a document typesetting system that is used 
      widely for documentation. It has fascilities for source code listing and beautifying. \LaTeX can
      easily merge source code into documentation by line number or arbitrary flag in case line numbers
      fluctuate.\xml{br/}
      \xml{br/}
      This Infoshare was written in \LaTeX. The overview.html and package.html files mentioned on the 
      previous slide were generated by \LaTeX
    \end{s5slide}
  \end{ifhtml}
  \W \end{s5presentation}
\end{document}