\documentclass[12pt]{report}
\author{Leo Przybylski}
\usepackage{graphicx}
\usepackage{listings}
\usepackage{color}
\usepackage{hyperref}
\usepackage{hyperlatex}
\setcounter{htmldepth}{0}

\definecolor{DarkBlue}{rgb}{0,0,0.55}
\definecolor{DarkGreen}{rgb}{0,0.4,0}
\definecolor{Purple}{rgb}{0.5,0,0.5}

\begin{ifhtml}
\newcommand{\sf}[1]{\xml{span style="font-family: sans-serif;"}#1\xml{/span}}
\newcommand{\rm}[1]{\xml{span style="font-family: serif;"}#1\xml{/span}}
\newcommand{\bf}[1]{\xml{span style="font-weight: bold;"}#1\xml{/span}}
\newcommand{\href}[2]{\xml{a href="#1"}#2\xml{/a}}
\newcommand{\HlxStyleSheet}{
  \begin{rawxml}
    <!-- metadata -->
    <meta name="generator" content="S5" />
    <meta name="version" content="S5 1.1" />
    <meta name="presdate" content="\end{rawxml}\HlxDate\begin{rawxml}"/>
    <meta name="author" content="\end{rawxml}\HlxAuthor\begin{rawxml}"/>
    <meta name="company" content="Leosandbox " />
    <!-- configuration parameters -->
    <meta name="defaultView" content="slideshow" />
    <meta name="controlVis" content="hidden" />
    <!-- style sheet links -->
    <link rel="stylesheet" href="ui/kuali/slides.css" type="text/css" media="projection" id="slideProj" />
    <link rel="stylesheet" href="ui/kuali/outline.css" type="text/css" media="screen" id="outlineStyle" />
    <link rel="stylesheet" href="ui/kuali/print.css" type="text/css" media="print" id="slidePrint" />
    <link rel="stylesheet" href="ui/kuali/opera.css" type="text/css" media="projection" id="operaFix" />
    <script src="ui/kuali/slides.js" type="text/javascript"></script>
  \end{rawxml}
}

\newcommand{\maketitle}{
    \xml{div class="slide"}
    \EmptyP{\HlxTitleP}{\HlxBlk\xml{h1}\HlxTitle\xml{/h1}
    \EmptyP{\HlxAuthorP}{\xml{h2}\HlxAuthor\xml{/h2}}{}
    \EmptyP{\HlxDate}{\xml{h3}\HlxDate\xml{/h3}}{}
    \xml{/div}
    }{}}

\end{ifhtml}

\newenvironment{s5presentation}{\endpar%
  \HlxBlk\begin{rawxml}
  <div class="header">
    <div class="header_l">
      <div class="header_r">
      &nbsp;
      </div>
    </div>
  </div>
  <div class="content">
    <div class="content_l">
      <div class="content_r">
        <img src="ui/kuali/blank.gif" id="filler" border="0"/>
<div class="layout">
  <div id="controls"><!-- DO NOT EDIT --></div>
  <div id="currentSlide"><!-- DO NOT EDIT --></div>
  <div id="header">
    <img style="position: relative; top:30px; left: 18px;" src="ui/kuali/logo.png" />
  </div>
  <div id="footer">
    <div class="footer_l">
      <div class="footer_r">
        <h1/>
        <h2>\end{rawxml}\HlxTitle\begin{rawxml}</h2>
      </div>
    </div>
  </div>
</div>
    </div>
  </div>
 </div>
<div class="presentation">
    \end{rawxml}}{\HlxBlk\xml{/div}}
\newenvironment{s5slide}{\xml{div class="slide"}}{\xml{/div}}
\newcommand{\fulltitle}[1]{\title{#1}\htmlonly{\htmltitle{#1}}}

\fulltitle{Kuali Research Administration Logging Specification}
\W \subtitle{}

\W \affiliation{University of Arizona}

\begin{document}
\maketitle
\tableofcontents

\abstract{A specification that can be used by software developers to create a work of software specific
to the purpose outlined in the requirements. This specification addresses logging within the KRA project. 
Topics addressed are \begin{itemize}
  \item What do we need to log?
  \item When do we need to log?
  \item Where do we need to log?
  \item How do we want to log? 
  \item Layered approach to logging
  \item AOP Advice for logging
\end{itemize}

The goal of this document is to transmit that information to the reader (a software developer.) This document can
in some ways be considered a hybrid of functional and technical specification.}

\section{Scope}
Implementation of a logging strategy using patterns of \emph{Chain of Responsibility} and AOP. To obtain a
consistent approach to logging that is flexible, modular, and non-invasive, a strategy
is described through layers.

\subsection{Filter Layer}
Filters use the aforementioned \emph{Chain of Responsibility} to link filters together in order of execution. The J2EE
Servlet API provides interfaces for this which is what will be used to build the implementation. The use of Filters and 
Listeners are applicable.

\subsubsection{State Logging}
An implementation of filters to log in detail application and session state information for trace and debugging.

\subsubsection{Performance Logging}
An implementation of filters to log runtime information about the application and session that is useful for exposing
memory leaks or other resource related issues regarding overall application performance.

\subsection{Service Layer Logging}
Spring provides a facility for trace logging of services. Wes Price created an implementation that was extracted into
Kuali Rice called the MethodLoggingInterceptor. Reusing this source will enable trace level logging information for 
services which is useful in debugging.

\subsection{Domain Model Logging}
Currently not officially scoped

\subsection{Exception Logging}
Exception Logging is outside the scope of this document. Please read 

https://test.kuali.org/confluence/display/KULRICE/Exception+handling+and+incident+reporting

This may be revisited at a later time.

\section{Referenced Documents}
Exception Handling
 
https://test.kuali.org/confluence/display/KULRICE/Exception+handling+and+incident+reporting

\section{Requirements}
\subsection{External Interface Requirements}
The following is an outline of API's, libraries, components, and protocols used by the software being implemented. If the software
cannot be implemented within the guidelines outlined here, this specification should be revised.

\subsubsection{Libraries}
\begin{itemize}
\item Log4J
\item Spring
\item OpenSymphony Clickstream (waiting to see if this is part of KTC approved API list)
\end{itemize}

\subsubsection{Application Programming Interfaces}
\begin{itemize}
  \item J2EE Servlet API v2.3
\end{itemize}

\subsection{Software Applications}
\begin{itemize}
  \item Jetty
  \item Tomcat v5.5
\end{itemize}

\subsection{Design and Implementation Constraints}
Developers may be able to avoid continuous \verb|isDebugEnabled()| checks with an interface to easily defer string
concatenation and creation to prevent needless creation and destruction of strings. The interface provides easily substitutable,
non-intrusive wrappers for log levels described in Other Requirements.

\subsubsection{Layered Logging Approach}
Filters and ServletListeners are used for Session and Application state logging and for performance logging. 

\begin{description}
\item [Session and Application state] is logged per request. The state consists of:
\begin{itemize}
  \item All request headers (headers sent from the client)
  \item The target URI
  \item All request parameters (POST and GET)
  \item The client IP address is logged
\end{itemize}
Further information may be logged in the future regarding session state. See the \emph{Notes} section regarding 
this.

The following parts should be logged at the *INFO* level:
\begin{itemize}
  \item Current logged in user name
  \item Current date/time
  \item The action or whatever best describes what the user was trying to do
\end{itemize}

All other parts of state logging should be logged at *DEBUG* level.

\item [Performance data] is logged per request. The \textbf{INFO} log level is used. The data is captured immediately before a request and after a request to measure the difference
and report it to the log. The performance measuring filter is as close to the end of the FilterChain as possible in order to have
the least amount of impact on performance measurements. Performance data consists of:
\begin{itemize}
  \item Request runtime in milliseconds
  \item JVM Memory consumption
  \item Available memory
\end{itemize}

\item [Service tracing] is implemented through Spring configuration using the \verb|org.kuali.core.util.spring.MethodLoggingInterceptor|
class. Tracing reports invocation details of each Spring service call. Invocation details consist of:
\begin{itemize}
  \item Input parameters including types, values, and default values
  \item Return value(s)
  \item If an exception is thrown, this is acknowledged in the trace. Exception details are logged.
\end{itemize}

Service tracing is turned on per service. Service tracing may be activated for all services.
\end{description}

\subsubsection{Configuration Constraints}
The following configuration files may be effected:
\begin{description}
  \item[log4j.properties] log formats, appenders, log levels, et al... may need to be modified. 
  \item[SpringBeans.xml] For AOP MethodLoggingInterceptor configuration
  \item[web.xml] Filters have to be declared within the web.xml file
\end{description}

\subsubsection{Allowable Refactoring}
It may be required to re-implement the \verb|MethodLoggingInterceptor|.

\subsection{User Documentation}
In this case, the user is a developer. Developers will use logs and implementation to provide logging in their functionality.
The javadoc guidelines followed can be found at \verb|http://java.sun.com/j2se/javadoc/writingdoccomments/#doccommentcheckingtool|, 
and \verb|https://test.kuali.org/confluence/display/KRACOEUS/Coding+Standards|

Developers may also be required to interpret log information. Provide documentation on how to interpret logs. When log formats
change, update the documentation to reflect the change in intepretation.

Since AOP Advice for tracing service calls must be manually turned on by a developer, it is thorougly documented with examples
exactly how to turn tracing on for a specific service by editing the Spring configuration.

\subsection{Performance Requirements}
Logging statements should have little to no impact on the application. Minimize impact as much as possible when logging.

It is currently unknown what can be considered impact to the application. This will be determined at a later time.

\subsection{Security Requirements}
Confidential or sensitive information is never logged. This is a requirement for all logging and all log messages. No log should ever
expose the following:
\begin{itemize}
  \item Passwords
  \item Social security numbers
  \item Personal identification numbers
  \item Tax identification numbers
  \item Financial information like account numbers 
  \item Any personal identifiable information of individuals under the age of 13
\end{itemize}

The logging implementation must not interfere or intercede on the login/logout process. For example, when using a CachingFilter, cached pages
only available to authenticated users should not be responded after the session has expired.

\subsection{Software Quality Factors}
Tests are provided to show that the logging implementation works and is not functioning outside the boundaries of this specification. It is not
important to test that log messages are printed to the appropriate output. Filters are proven to not greatly impact runtime by consuming
unnecessary resources like memory or cpu usage. Tests show that the logging implementation does not in any way effect the functionality of the
application. Test show that there are no usability issues caused from the logging implementation.

\subsection{Other Requirements}
These are the definitions for the applicable log levels of the logging strategy implementation.
\subsubsection{DEBUG}
\begin{itemize}
    \item Entry and exit points to other methods are logged with this level (utility, dao, etc.,) including arguments passed and returned values where appropriate / branches in code.
    \item If log of params or return values will take more than a few lines
\end{itemize}

\subsubsection{INFO}
\begin{itemize}
\item Used for entry an exit points to service methods. 
\item Core information (not in the sense of everything in the ns) - startup and shutdown points / start and end of request and function requested
\end{itemize}

\subsubsection{WARN}
\begin{itemize}
  \item Used for unexpected situations where throwing an error is not appropriate. if the situation occurs frequently, it is not unexpected.
\end{itemize}

\subsubsection{ERROR}
\begin{itemize}
  \item Used for global exception handling.
\end{itemize}

\subsubsection{FATAL}
\begin{itemize}
  \item When scenarios appear like a database cannot be connected to or workflow is inaccessible, this log level is used.
  \item MethodLoggingInterceptor requires the use of FATAL. This may need to be reimplemented.
\end{itemize}

\section{Requirements Traceability}
This is where it is described how requirements trace back to source code. 

\subsection{Expected Modified Classes}
\subsection{Expected New Classes}

\section{Analysis Patterns/Models}
\subsection{Layered Approach to Logging}
The diagram below illustrates where in the KRA application the layers are for logging.
\includegraphics[scale=0.5,bb=40 0 100 600]{LoggingByLayers.eps}
\newpage
\subsection{Chain of Responsibility}
\begin{quote}
Avoid coupling the sender of a request to its receiver by giving more than one object a chance to handle
the request. Chain the receiving objects and pass the request along the chain until an object handles it. -- Gang of Four\end{quote}

The diagram below illustrates how KRA implements this pattern with filters.

\includegraphics[scale=0.45,bb=40 0 100 350]{ChainOfResponsibility.eps}

\section{Notes}
\subsection{Session State. What gets logged?}
Bryan is looking into this. Some options given were:
\begin{itemize}
  \item Everything except the Form 
  \item Everything in the Form except the document
  \item Maybe this is only the case for specific url patterns?
  \item Maybe just forget session state? 
\end{itemize}

\end{document}
